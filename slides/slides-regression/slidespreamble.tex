%\useoutertheme[glossy]{wuerzburg}
\useinnertheme[shadow,outline]{chamfered}
%\usecolortheme{shark}
\usecolortheme{beaver}
\beamertemplatenavigationsymbolsempty

\usefonttheme{professionalfonts}
\let\digamma\relax
\usepackage[scale=0.85,stdmathitalics=true,romanfamily=casual]{lucimatx}
\usefonttheme[stillsansseriftext]{serif}

\usepackage{bm}
\usepackage{amsmath}

\usepackage{fancyvrb}

%% Fancy syntax coloring via pygments
%\usepackage{minted}
%\definecolor{bg}{rgb}{0.95,0.95,0.95}
%\usemintedstyle{borland}


% \newenvironment{Rcode}
% {\VerbatimEnvironment
%  \begin{minted}[fontsize=\scriptsize,baselinestretch=1]{r}}%
% {\end{minted}}

% \newenvironment{Pcode}
% {\VerbatimEnvironment
% \begin{minted}[fontsize=\scriptsize,baselinestretch=1]{python}}%
% {\end{minted}}

% \newenvironment{Code}[1]
% {\VerbatimEnvironment
 % \begin{minted}[fontsize=\scriptsize,baselinestretch=1]{#1}}%
% {\end{minted}}


\usepackage{textfit} % commands \scaletoheight{height}{text} and \scaletowidth{width}{text}

\usepackage{tikz}

\usepackage{tcolorbox}

\usepackage{subcaption}

\newtheorem{Alert}{Alert}
\newtheorem{Highlight}{Highlight}

\newcommand{\Species}[1]{{\rmfamily \itshape #1}}
\newcommand{\Real}{\ensuremath{\mathbb{R}}}
\newcommand{\RealN}{\ensuremath{\mathbb{R}^n}}
\newcommand{\RealP}{\ensuremath{\mathbb{R}^p}}
\newcommand{\Mtx}[1]{\ensuremath{\bm{#1}}}
\newcommand{\Inv}[1]{\ensuremath{#1^{-1}}}
\newcommand{\InvMtx}[1]{\ensuremath{\bm{#1}^{-1}}}
\newcommand{\Red}[1]{\textcolor{red}{#1}}
\newcommand{\PsInv}[1]{\ensuremath{\bm{#1}^{+}}}
\DeclareMathOperator{\cov}{cov}
\DeclareMathOperator{\corr}{corr}

\usepackage{booktabs}



% --- Macro \xvec
% From a tex.stackexchange.com answer by Todd Lehman
% http://tex.stackexchange.com/questions/44017/dot-notation-for-derivative-of-a-vector
\makeatletter
\newlength\xvec@height%
\newlength\xvec@depth%
\newlength\xvec@width%
\newcommand{\xvec}[2][]{%
  \ifmmode%
    \settoheight{\xvec@height}{$#2$}%
    \settodepth{\xvec@depth}{$#2$}%
    \settowidth{\xvec@width}{$#2$}%
  \else%
    \settoheight{\xvec@height}{#2}%
    \settodepth{\xvec@depth}{#2}%
    \settowidth{\xvec@width}{#2}%
  \fi%
  \def\xvec@arg{#1}%
  \def\xvec@dd{:}%
  \def\xvec@d{.}%
  \raisebox{.2ex}{\raisebox{\xvec@height}{\rlap{%
    \kern.05em%  (Because left edge of drawing is at .05em)
    \begin{tikzpicture}[scale=1]
    \pgfsetroundcap
    \draw (.05em,0)--(\xvec@width-.05em,0);
    \draw (\xvec@width-.05em,0)--(\xvec@width-.15em, .075em);
    \draw (\xvec@width-.05em,0)--(\xvec@width-.15em,-.075em);
    \ifx\xvec@arg\xvec@d%
      \fill(\xvec@width*.45,.5ex) circle (.5pt);%
    \else\ifx\xvec@arg\xvec@dd%
      \fill(\xvec@width*.30,.5ex) circle (.5pt);%
      \fill(\xvec@width*.65,.5ex) circle (.5pt);%
    \fi\fi%
    \end{tikzpicture}%
  }}}%
  #2%
}
\makeatother

% --- Override \vec with an invocation of \xvec.
\let\stdvec\vec
\renewcommand{\vec}[1]{\xvec[]{\bm{#1}}}
% --- Define \dvec and \ddvec for dotted and double-dotted vectors.
\newcommand{\dvec}[1]{\xvec[.]{#1}}
\newcommand{\ddvec}[1]{\xvec[:]{#1}}


\usepackage{pifont}
\newcommand{\weblink}{\ding{43}}  % hand with pointing finger

\definecolor{links}{HTML}{2A1B81}
\hypersetup{colorlinks,linkcolor=,urlcolor=magenta}